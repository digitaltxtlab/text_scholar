\documentclass[8pt]{beamer}
\usetheme{Pittsburgh}
	\usecolortheme{dove}
\usepackage[utf8]{inputenc}
\usepackage{amsmath}
\usepackage{amsfonts}
\usepackage{amssymb}


% for R code in box and w. text coloring
\usepackage{color}
\usepackage{fancyvrb}
\fvset{frame=single,framesep=1mm,fontfamily=courier,fontsize=\scriptsize,numbers=left,framerule=.3mm,numbersep=1mm,commandchars=\\\{\}}

%% code in box and w. text coloring
\RequirePackage{color}
\RequirePackage{fancyvrb}
\fvset{frame=single,framesep=1mm,fontfamily=courier,fontsize=\scriptsize,numbers=left,framerule=.3mm,numbersep=1mm,commandchars=\\\{\}}

% footnote size
\let\oldfootnotesize\footnotesize
\renewcommand*{\footnotesize}{\oldfootnotesize\tiny}

% metadata
\author{K.L. Nielbo}
%\title{}
%\setbeamercovered{transparent} 
%\setbeamertemplate{navigation symbols}{} 
\addtobeamertemplate{navigation symbols}{}{%
    \usebeamerfont{footline}%
    \usebeamercolor[fg]{footline}%
    \hspace{1em}%
    \insertframenumber/\inserttotalframenumber
}
%\logo{\includegraphics[scale=.1]{/home/kln/Pictures/imc_logo.png}} 

% logo control
\newcommand{\nologo}{\setbeamertemplate{logo}{}} 
\logo{%
  \makebox[.98\paperwidth]{%
    \includegraphics[scale=.07]{/home/kln/Pictures/aulogo.jpg}%
    \hfill%
    \includegraphics[scale=.125]{/home/kln/Pictures/imclogo.png}%
  }%
}

\title{Digital Humanities \& Humanities Data}
\subtitle{}
\author{Kristoffer L Nielbo \\\texttt{kln@cas.au.dk}}
\date{}
\institute{DAI$\mid$IMC$\mid$AARHUS UNIVERSITY}

\begin{document}

%%%%% introduction
\begin{frame}
\titlepage
\end{frame}

\begin{frame}
	\begin{center}
		\textbf{does is make sense to talk about a unified DH anymore?}
	\end{center}
\end{frame}

\begin{frame}{}
\textbf{DIGITAL HUMANITIES$\mid$DH}\\
\noindent\rule{1cm}{0.4pt}\\
-- the \textit{\textcolor{purple}{conjunction}} of the digital and the humanities\\
-- \textcolor{purple}{redrawing boundary} lines traditional academic boundaries\\
-- expanding the audience and \textcolor{purple}{social impact of scholarship} in the humanities\\
-- developing \textcolor{purple}{new forms of inquiry} and knowledge production\\
-- training \textcolor{purple}{future generations} of humanists through hands-on, project-based learning\\
-- increase \textcolor{purple}{visibility} of humanistic inquiry\\
\medskip
\textit{DH has gone through several distinct transformations during the last 3 decades}\\
\noindent\rule{2cm}{0.4pt}\\
\textbf{HUMANITIES COMPUTING} -- computational tools in traditional humanities\\
\textbf{DIGITAL HUMANTIES 2.0} -- digital culture, activism and critique\\
\textbf{HUMANITIES DATA} -- data-intensive research in SSH domains\\
\noindent\rule{4cm}{0.4pt}\\
\end{frame}


%%%%% what happend

\begin{frame}{}
\begin{columns}
\begin{column}{0.5\textwidth}
	\begin{center}
		\includegraphics[scale=.25]{/home/kln/Pictures/technologies_10_nologo.jpg}     	
     \end{center}
\end{column}
\begin{column}{0.5\textwidth}
    \begin{center}
     	\includegraphics[scale=.325]{/home/kln/Pictures/data_explosion.png}
    \end{center}
\end{column}
\end{columns}
\bigskip 
\bigskip
\small
\begin{quote}
`Digitization and digital media have generated a rapid proliferation of data that is unprecedented in the history of man. This \textcolor{purple}{\textbf{digital surge}} is transforming knowledge discovery and understanding in every domain of human inquiry. Digital research, computing, data management, and data-intensive methods will therefore become integral parts of internationally-leading research in the humanities and arts.'
\end{quote}
\begin{flushright}
(Digital Arts Strategy 2016)
\end{flushright}
\end{frame}

%%%%% example
\begin{frame}%{eHumanities forskningseksempel}
	\begin{center}
		\includegraphics[scale=.3]{/home/kln/Pictures/ts_example.png}
	\end{center}
	\small \textbf{humanities data example}
	 -- humanistic domain knowledge (history og medie studies), data engineering, programming, statistics \& visualisation\\
\end{frame}

%%%%% more than one word on data
\begin{frame}
	\begin{center}
		\textbf{what is so new about DATA (or we always had data)?}
	\end{center}
\end{frame}

\begin{frame}{}%data $\neq$ knowledge}
	\begin{center}
		\includegraphics[scale=.3]{/home/kln/Pictures/data_knowledge.png}
	\end{center}
\end{frame}
\begin{frame}{}
	\begin{center}
		\includegraphics[scale=.3]{/home/kln/Pictures/data_knowledge_data.png}
	\end{center}
\end{frame}
\begin{frame}{}
	\begin{center}
		\includegraphics[scale=.3]{/home/kln/Pictures/data_knowledge_info.png}
	\end{center}
\end{frame}
\begin{frame}
	\begin{center}
		\textbf{DATA modifies our workflow?}
	\end{center}
\end{frame}
\begin{frame}
	\begin{center}
		\includegraphics[scale=.3]{/home/kln/Documents/education/tm_R/classes/intro_p4/f1.PNG}
	\end{center}
\end{frame}
\begin{frame}
	\begin{center}
		\includegraphics[scale=.3]{/home/kln/Documents/education/tm_R/classes/intro_p4/f2.PNG}
	\end{center}
\end{frame}
\begin{frame}
	\begin{center}
		\includegraphics[scale=.3]{/home/kln/Documents/education/tm_R/classes/intro_p4/f3.PNG}
	\end{center}
\end{frame}
\begin{frame}
	\begin{center}
		\includegraphics[scale=.3]{/home/kln/Documents/education/tm_R/classes/intro_p4/f4.PNG}
	\end{center}
\end{frame}
\begin{frame}
	\begin{center}
		\includegraphics[scale=.3]{/home/kln/Documents/education/tm_R/classes/intro_p4/f5.PNG}
	\end{center}
\end{frame}

%%%%% 3 techniques
\begin{frame}
	\begin{center}
		\textbf{examples of data-intensive methods in DH}
	\end{center}
\end{frame}

%%% sentiment analysis
\begin{frame}
\textbf{PROBLEM \#1}\\
\noindent\rule{1cm}{0.4pt}\\
a document is an ordered set of words that (at least in part) expresses the \textcolor{purple}{cognitive and affective states} of the author\\
\medskip
we want a \textcolor{purple}{automatized method} that transfers psychological scales to documents and maintain validity and reliablility\\
\medskip 
preferably, the method should be \textcolor{purple}{scalable} both in terms of quantity and context\\
\end{frame}

\begin{frame}[fragile]
a \textcolor{purple}{dictionary} can identify keywords in a collection of documents and apply a sentiment function\\
\medskip
\begin{Verbatim}
'Did \textcolor{red}{Crooked} \textcolor{red}{Hillary} help \textcolor{red}{disgusting} (check \textcolor{red}{out} \textcolor{green}{sex} tape and past) Alicia M become a U.S. \textcolor{green}{citizen} 
so she could use her in the debate?'
 
\textbf{Positive} sex, citizen
\textbf{Negative} crooked, hillary, disgusting, out
\textbf{Sentiment Score} (2+1) + (-2-1-3-1) = -4 
\textbf{Sentiment Polarity} Negative
\textbf{Overall Score} Sum of all sentence scores
\end{Verbatim}
\medskip
a sentiment vector is simply a vector of keyword frequencies weighted by sentiment scores\\
\end{frame}

\begin{frame}
\textbf{sentiment analysis} a set of methods for extracting the (primarily) affective components from unstructured data\\ 
\medskip
utilize existing dictionaries to avoid tedious manual coding and validation*\\ 
\medskip
three general approaches:\\
\begin{itemize}
\item[-] \textcolor{purple}{dictionary-based methods} (word counting)
\item[-] supervised learning (ML)
\item[-] unsupervised learning (ML)
\end{itemize}
\end{frame}


%%% topic modeling

\begin{frame}
\textbf{PROBLEM \#2}\\
a document is a structured* or \textcolor{purple}{non-random collection of words}, but who or what is the structuring agent?\\
\medskip
to \textcolor{purple}{avoid manual modeling}, we want method that we can throw ++documents at and then it will sort things out\\
\medskip
preferably, the output should exhibit some degree of similarity with human text comprehension\\
\end{frame}


\begin{frame}
\medskip
we can extract latent topics that generated the documents by reverse engineering the process  $words~ \&~doc \Rightarrow topics$\\
\medskip
decompose an n-by-d word document matrix into two matrices: a n-by-k \textcolor{red}{word topic matrix} and an k-by-d  \textcolor{green}{topic document matrix}\\
	\begin{center}
		\includegraphics[scale=1.5]{/home/kln/Pictures/matrixfact.png}
	\end{center}
\medskip
to ``generate'' a document, choose a distribution over topics, sample a topic (k-by-d matrix), then a word from this topic (n-by-k matrix) and repeat
\end{frame}

\begin{frame}{}
	\begin{columns}
		\begin{column}{.35\textwidth}
			\includegraphics[width=1\textwidth]{/home/kln/Pictures/doc_dist.png}
			
			\includegraphics[width=1\textwidth]{/home/kln/Pictures/topic.png}
		\end{column}
		\begin{column}{.55\textwidth}			 
			\includegraphics[width = 1\textwidth]{/home/kln/Pictures/document.png}
		\end{column}
	\end{columns}
\end{frame}



% word embedding
\begin{frame}
\textbf{PROBLEM \#3}\\
word meaning is \textcolor{purple}{dependent on word context} and words in similar contexts have similar meanings\\  
\medskip
to increase computational efficiency and obtain distributed representations for words, we want a methods that can learn every association from a large set of documents without human interference\\ 
\medskip
preferably, the method should \textcolor{purple}{emulate human concept learning}\\
\end{frame}



\begin{frame}%{Word embedding - the new black}
we can train an \textcolor{purple}{Artificial Neural Network} to predict a word given its context (CBOW) or predict a context given a word (Skip-gram)\\
\medskip
\begin{center}
	\includegraphics[scale=.25]{/home/kln/Pictures/word_embed_arch.png}
\end{center}
the word representations (word vectors) used by the ANN to solve the task will \textcolor{purple}{reflect important semantic features} 
\end{frame}

\begin{frame}
	\begin{center}
		\includegraphics[scale=.35]{/home/kln/Pictures/tsne_main_embed.png}
	\end{center}
\end{frame}




%%%%% a word on tools
\begin{frame}
	\begin{center}
		\textbf{available tools}
	\end{center}
\end{frame}

\begin{frame}{}
\begin{center}
\includegraphics[scale=.75]{/home/kln/Pictures/bigDataEcoSys.png} 
\end{center}
\end{frame}

\begin{frame}
\begin{columns}
\begin{column}{0.5\textwidth}
   \begin{center}
		\includegraphics[width=1.2\textwidth]{/home/kln/Pictures/pythonLogo.png}   
   \end{center}
\end{column}
\begin{column}{0.5\textwidth}
    \begin{center}
		\includegraphics[width=0.5\textwidth]{/home/kln/Pictures/Rlogo.png}    
    \end{center}
\end{column}
\end{columns}
\end{frame}

\begin{frame}
\begin{columns}
\begin{column}{0.5\textwidth}
   \begin{center}
    \includegraphics[width=1\textwidth]{/home/kln/Pictures/rapidminer.png}\\
    \bigskip
    \includegraphics[width=1\textwidth]{/home/kln/Pictures/tableau.png}
    \end{center}
\end{column}
\begin{column}{0.5\textwidth}
    \begin{center}
    \includegraphics[width=.5\textwidth]{/home/kln/Pictures/antconc.jpg}\\
    \bigskip
    \includegraphics[width=.8\textwidth]{/home/kln/Pictures/voyant.png}
    \end{center}
\end{column}
\end{columns}
\end{frame}

\end{document}